\begin{abstract}
Uno de los problemas a los que se enfrentan los estudiosos de la gesti�n de memoria din�mica es la falta de documentaci�n relativa al modo en que se gestiona la memoria din�mica en el sistema operativo GNU/Linux. La resoluci�n de este problema es relevante ya que deber�a permitir asentar las bases para el descubrimiento de nuevas vulnerabilidades en el algoritmo encargado de realizar dicha gesti�n. \bigskip

El objetivo de esta investigaci�n es presentar un marco te�rico sobre el cual fundamentar el estudio de las vulnerabilidades en el algoritmo \textit{ptmalloc2} tal y como est� implementado en la \textit{glibc} 2.12.1 y as� reducir las dificultades que presenta la falta de documentaci�n que se ha mencionado. \\
A tal efecto, la primera fase de esta investigaci�n se bas� en realizar un estudio te�rico del funcionamiento del algoritmo \textit{ptmalloc2}. Posteriormente se realiz� un estudio sobre las vulnerabilidades publicadas de dicho algoritmo, focaliz�ndose en las vulnerabilidades m�s significativas del pasado hasta llegar al estado del arte, de modo que una vez realizado el estudio te�rico de dichas vulnerabilidades se prosigui� con la comprobaci�n emp�rica de la funcionalidad de las t�cnicas establecidas para la explotaci�n de una de las vulnerabilidades m�s relevantes. \bigskip

En el proceso de esta investigaci�n se estableci� cual era el estado del arte real en cuanto a las t�cnicas utilizadas para aprovecharse de las vulnerabilidades del algoritmo \textit{ptmalloc2}. Tambi�n se present� una modificaci�n de una de las t�cnicas existentes en el pasado pero ahora obsoleta, con la cual es posible atacar el algoritmo \textit{ptmalloc2} de manera satisfactoria. \bigskip

Los resultados obtenidos permiten dar un nuevo enfoque a las t�cnicas utilizadas para explotar las vulnerabilidades en la gesti�n de la memoria din�mica, presentando una modificaci�n de una t�cnica utilizada en el pasado con el beneficio de tener menos requisitos que las t�cnicas utilizadas en la actualidad.

\vspace{70pt}

\textbf{Keywords:} \textit{dynamic allocation, heap overflow, heap exploitation, me\-mory corruption, ptmalloc, glibc, unlink, malloc maleficarum}

\end{abstract}