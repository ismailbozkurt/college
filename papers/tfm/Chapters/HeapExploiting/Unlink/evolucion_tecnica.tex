\subsection{Evoluci�n de la t�cnica}

Por otro lado, ser�a correcto justificar porqu� las t�cnicas detalladas en los art�culos citados anteriormente ya no funcionan. \\
La metodolog�a que se explicaba en ellos era muy parecida a la detallada en el cap�tulo anterior, sin embargo, a parte de otras diferencias m�s sutiles, en esos art�culos en vez de sobrescribir el campo |size| del segundo fragmento con un -15 (a efectos pr�cticos el tama�o es de -16 bytes) lo sobrescrib�an con un -4, en hexadecimal, 0xfffffffc. \\
Actualmente, si se utiliza dicho valor, la macro \textit{unlink} se ejecuta en el primer |free()|, sin embargo, cuando se est� liberando el segundo fragmento de memoria, el que contiene el -4 en el campo |size|, el programa termina su ejecuci�n recibiendo un |SIGSEGV|, o sea, realizando una violaci�n de segmento cuando se ejecuta la macro |arena_for_chunk()| en [malloc:3405]. \bigskip

Esto se debe a que el valor -4, en hexadecimal es 0xfc lo que en binario es 1111 0110. El segundo conjunto de 4 bits contiene el bit |NON_MAIN_ARENA| a 1. Este dato es relevante debido a que la macro |arena_for_chunk()| se define del siguiente modo: \bigskip

\lstset{language=C, caption=Macro arena\_for\_chunk() , label=code:arena_for_chunk}
\begin{lstlisting}
#define arena_for_chunk(ptr) \
 (chunk_non_main_arena(ptr) ? heap_for_ptr(ptr)->ar_ptr : &main_arena)
\end{lstlisting}

Se comprueba si el fragmento de memoria pertenece al arena principal a partir de la macro |chunk_non_main_arena()|, definida como: \bigskip

\lstset{language=C, caption=Macro chunk\_non\_main\_arena() , label=code:chunk_non_main_arena}
\begin{lstlisting}
/* size field is or'ed with NON_MAIN_ARENA if the chunk was obtained
   from a non-main arena.  This is only set immediately before handing
   the chunk to the user, if necessary.  */
#define NON_MAIN_ARENA 0x4

/* check for chunk from non-main arena */
#define chunk_non_main_arena(p) ((p)->size & NON_MAIN_ARENA)
\end{lstlisting}

Como se puede ver, la macro devolver� un valor diferente a uno si el campo |size| del fragmento de memoria tiene el bit |NON_MAIN_ARENA| a 1. \\
Tal y como ya se ha dicho, si el campo size contiene un -4, el bit |NON_MAIN_ARENA|, que en binario es 0100, estar� a uno con lo que la condici�n ser� positiva y se ejecutar� la macro |heap_for_ptr()| en vez de devolver la direcci�n del arena principal tal y como deber�a ocurrir. \\
Acto seguido, cuando se obtiene el puntero al arena a trav�s de dicha macro, se intenta obtener el campo |size|, pero debido a que dicho puntero no es correcto, se acaba incurriendo en una violaci�n de segmento. \bigskip

La �nica explicaci�n l�gica a este error publicado en los art�culos citados es que en la �poca en la que se publicaron el algoritmo \textit{ptmalloc} no implementaba el uso de diferentes arenas sino que siempre operaba sobre el mismo arena. Evidentemente, con la introducci�n del uso de m�s de un arena la t�cnica se volvi� obsoleta tal y como se demuestra con el C�digo \ref{out:pof_obsolete}. \bigskip

\begin{listing}[style=consola, caption=Ejecuci�n con el campo size a -4, label=out:pof_obsolete]
newlog@ubuntu:~/Documents/TFM/Heap/heap_exploiting/codes/unlink/ptmalloc2_test$ gdb -q pof_without_null_bytes
Leyendo s�mbolos desde /home/newlog/Documents/TFM/Heap/heap_exploiting/codes/unlink/ptmalloc2_test/pof_without_null_bytes...hecho.
(gdb) r
Starting program: /home/newlog/Documents/TFM/Heap/heap_exploiting/codes/unlink/ptmalloc2_test/pof_without_null_bytes 

Program received signal SIGSEGV, Segmentation fault.
0x0013087e in free (mem=0x804b210) at malloc.c:3405
3405	  ar_ptr = arena_for_chunk(p);
(gdb) x/3x 0x804b208
0x804b208:	0x40404041	0xfffffffc	0x08049f10
(gdb) 
\end{listing}

Como se puede ver, el puntero |mem| apunta a la direcci�n 0x804b210. Si a dicha direcci�n se le restan 8 bytes se obtiene la direcci�n del fragmento de memoria |p|\footnote{El valor del puntero p ha sido obtimizado por gdb, por esta raz�n no se obtiene su contenido con x/3x p.}. Como se puede ver, el campo |prev_size| del fragmento contiene el valor 0x40404041, que se ha sobrescrito con el desbordameinto, y el campo size contiene el valor 0xfffffffc, -4. Debido a lo explicado, este valor hace que el programa termine al ejecutar la macro |arena_for_chunk|.